% !TeX root=../main.tex

\chapter{پیش گفتار}

\section{مقدمه}

سیستم‌های گفتگوی وظیفه‌گرا به عنوان یکی از مهم‌ترین ابزارهای تعامل انسان و ماشین، نقش اساسی در بهبود تجربه کاربری در حوزه‌هایی مانند سیستم‌های توصیه‌گر، خدمات مشتری و تجارت الکترونیک ایفا می‌کنند \cite{camilleri2024artificial}. این سیستم‌ها با درک پرسش‌های کاربران و ارائه پاسخ‌های متناسب، تجربه‌ای طبیعی‌تر و مؤثر را در تعاملات کاربران با سیستم فراهم می‌کنند 
\cite{bocklisch2024task}. 
با این حال، چالش‌های زیادی از جمله نیاز به تنظیم دقیق مدل‌های زبانی، شخصی‌سازی پاسخ‌ها بر اساس ترجیحات کاربران، مشکل شروع سرد برای کاربران جدید و نگرانی‌های مربوط به حریم خصوصی همچنان به عنوان موانع اصلی در مسیر پیشرفت این فناوری‌ها مطرح هستند 
\cite{yuan2023user}.

یکی از مهم‌ترین چالش‌های این حوزه، شخصی‌سازی پاسخ‌ها و انطباق‌پذیری سیستم با نیازهای خاص هر کاربر است. در بسیاری از روش‌های موجود، اطلاعات کاربر به صورت ایستا در سیستم ذخیره می‌شود، اما عدم تطبیق آن با تغییرات در رفتار و سلیقه کاربران باعث کاهش دقت توصیه‌ها و کاهش رضایت کاربر به مرور زمان می‌شود 
\cite{azzam2022model}. 

علاوه بر این، مشکل شروع سرد، یعنی ناتوانی سیستم در ارائه پاسخ‌های دقیق به کاربران جدید که سابقه‌ای در سیستم ندارند، یکی از مسائل جدی در سیستم‌های توصیه‌گر و گفتگوی وظیفه‌گرا است
 \cite{yuan2023user}. 
از سوی دیگر، حفظ حریم خصوصی کاربران یکی دیگر از چالش‌های کلیدی در این زمینه است؛ کاربران نیاز دارند که داده‌هایشان در صورت درخواست حذف شود، اما بسیاری از روش‌های سنتی قابلیت اجرایی مؤثری در این زمینه ندارند
 \cite{zhang2024right}.

در کنار این چالش‌ها، روش‌های تنظیم دقیق مدل‌های زبانی، اگرچه باعث بهبود عملکرد سیستم‌های گفتگوی وظیفه‌گرا می‌شوند، اما به دلیل نیاز به منابع محاسباتی بسیار بالا و زمان طولانی پردازش، محدودیت‌هایی را ایجاد می‌کنند 
\cite{kasahara2022building}. 
علاوه بر این، روش‌های سنتی تنظیم دقیق معمولاً نیازمند حجم وسیعی از داده‌های آموزش هستند که در بسیاری از موارد به‌راحتی در دسترس نیستند. به همین دلیل، در سال‌های اخیر استفاده از روش تنظیم سریع مورد توجه قرار گرفته است، چرا که این روش بدون نیاز به تغییر تمام پارامترهای مدل، امکان انطباق سریع‌تر و مؤثرتر را با دامنه‌های خاص فراهم می‌کند 
\cite{madotto2021few}.

این پژوهش با هدف ارائه یک چارچوب جدید برای بهبود سیستم‌های گفتگوی وظیفه‌گرا، ترکیبی از تکنیک‌های تنظیم سریع و نماگر‌سازی پویا را پیشنهاد می‌دهد. در این روش، ابتدا داده‌های مرتبط با کاربر از طریق ترکیب حلیل احساسات برچسب‌های کاربر، فیلتر مشارکتی مبتنی بر آیتم، و تحلیل بازخورد کاربر استخراج شده و سپس با استفاده از تکنیک‌های تنظیم سریع و با استفاده از قدرت مدل های زبانی برای پاسخ‌گویی دقیق‌تر و شخصی‌سازی‌شده تنظیم می‌شود 
\cite{elahi2023hybrid}. 
این سیستم همچنین امکان حذف داده‌های کاربر را مطابق با حق فراموشی فراهم می‌کند تا از حریم خصوصی کاربران محافظت شود 
\cite{zhang2024right}.

با توجه به این رویکرد نوآورانه، انتظار می‌رود که نتایج این تحقیق بتواند راه را برای پیشرفت‌های آینده در حوزه شخصی‌سازی سیستم‌های گفتگوی وظیفه‌گرا هموار کند و الگویی برای بهبود تعاملات کاربر-ماشین در حوزه‌های مختلف از جمله تجارت الکترونیک، خدمات مشتریان، و سیستم‌های یادگیری هوشمند فراهم آورد 
\cite{chen2023zero}.

\section{بیان مسئله}
تحقیقات موجود در حوزه سیستم‌های گفتگوی وظیفه‌گرا به رغم پیشرفت‌های چشمگیر، همچنان دارای کاستی‌هایی هستند:
\begin{itemize}
\item
نبود تمرکز بر حفظ حریم خصوصی کاربران: بسیاری از سیستم‌های موجود داده‌های کاربر را بدون امکان حذف یا اصلاح ذخیره می‌کنند، که می‌تواند منجر به نگرانی‌های امنیتی و اخلاقی شود.
\item
نماگر‌سازی ناکارآمد: روش‌های موجود در نماگر‌سازی کاربران اغلب تنها به استفاده از یک تکنیک محدود هستند و از روش هایی مانند بازخورد موثر کاربران و نماگر پویا برای به‌روزرسانی مدل‌ها استفاده نمی‌کنند.
\item
نیاز به منابع سخت‌افزاری قدرتمند: تنظیم دقیق مدل‌های زبانی به منابع عظیم سخت‌افزاری نیاز دارد که اجرای آن را برای بسیاری از کاربران و سازمان‌ها دشوار می‌کند.
\end{itemize}

روش‌های متعددی با استفاده از متدهای مختلف هم با و هم بدون استفاده از مدل‌های زبانی طی سال ‌های اخیر ارایه شده است اما آنها نیز دارای کاستی‌هایی در کار خود هستند:
\begin{itemize}
\item
نبود تنوع در روش‌های نماگر‌سازی: تحقیقات فعلی عمدتاً به یک رویکرد محدود می‌شوند، در حالی که ترکیب روش‌های مختلف می‌تواند به دقت بیشتری در شخصی‌سازی پاسخ‌ها منجر شود که در نهایت به بهبود سیستم و افزایش کیفیت مکالمه و رضایت کاربر منجر می شوند.
\item
مشکلات شروع سرد: سیستم‌های موجود در مواجهه با کاربران جدید یا داده‌های ناکافی، عملکرد ضعیفی دارند.
\item
عدم امکان تنظیم سریع: بیشتر تحقیقات به تنظیم دقیق مدل وابسته هستند که نه تنها زمان‌بر و پرهزینه است، بلکه انعطاف‌پذیری کمتری نسبت به تنظیم سریع دارد.
\item
عدم رعایت حق فراموشی: کاربران نمی‌توانند داده‌های خود را حذف یا به‌روزرسانی کنند، که این مسئله اعتماد به سیستم را کاهش می‌دهد.
\end{itemize}


\section{اهداف پژوهش}

اهداف پژوهش باید منعکس‌کننده چالش‌های اصلی، گپ‌های موجود در ادبیات، و راه‌حل‌هایی باشد که تحقیق حاضر ارائه می‌دهد. 

هدف از اين تحقیق ارائه روشی مبتنی بر استفاده از تکنیکهای تنظیم سريع به همراه پروفايل سازی کاربر جهت ارتقای سیستم های گفتگوی وظیفه گرا و بهبود آنها است.


\section{فرضیه‌های پژوهش}


فرضیه اصلی پژوهش به صورت زیر است.

استفاده از روش ترکیبی نماگر‌سازی کاربر به همراه تنظیم سریع در سیستم‌های گفتگوی وظیفه‌گرا می‌تواند به طور موثری منجر به ارتقای عملکرد آن‌ها در پاسخگویی به نیازهای کاربران گردد.

این فرضیه به صورت مستقیم به بررسی نحوه استفاده از نماگر کاربر و تنظیم سریع برای بهبود معیارهایی چون رضایت کاربر، تعامل، و دقت پاسخ‌ها می‌پردازد. ترکیب این دو روش، با رویکردی جامع‌تر، مشکلاتی نظیر شروع سرد، حریم خصوصی و تعاملات بلندمدت کاربر را هدف قرار می‌دهد.


\section{سوال‌های پژوهش}

در این بخش، سوالات پژوهش بیان شده است. این فرضیات مبتنی بر شکاف‌های موجود در ادبیات پژوهشی و نیازهای سیستم‌های گفتگوی وظیفه‌گرا تدوین شده‌اند.

\begin{enumerate}
\item
استفاده از تنظیم سریع به چه صورت می‌تواند منجر به ارتقای سیستم گفتگوی وظیفه‌گرا شود؟

\item
استفاده از داده‌های نماگر کاربر در سیستم‌های گفتگوی وظیفه‌گرا از چه راه‌هایی بر رفتار و تعامل کاربر تأثیر می‌گذارد و چگونه می‌توان از این اطلاعات برای بهبود عملکرد کلی سیستم گفتگو استفاده کرد؟

\item
چه ویژگی‌ها و شاخص‌هایی باید در نماگر‌سازی کاربر در سیستم‌های گفتگوی وظیفه‌گرا مورد توجه قرار گیرند؟
\end{enumerate}

\section{نوآوری‌ها و سهم پژوهش}
در این بخش، نوآوری‌ها و سهم علمی پژوهش حاضر با توجه به دستاوردهای اصلی و ایده‌های منحصربه‌فرد آن ارائه می‌شود.
\begin{enumerate}
\item
معرفی رویکرد ترکیبی تنظیم سریع و نماگر‌سازی کاربر

یکی از مهم‌ترین نوآوری‌های پژوهش حاضر، معرفی یک رویکرد ترکیبی است که از تنظیم سریع و نماگر‌سازی پیشرفته کاربر بهره می‌برد. این رویکرد نشان می‌دهد که چگونه تلفیق داده‌های شخصی‌سازی‌شده با تنظیم سریع می‌تواند به بهبود پاسخ‌دهی و رضایت کاربر منجر شود. در این راستا، یک مدل مفهومی برای ادغام داده‌های کاربر و تنظیم سریع ارائه شده است که قابلیت استفاده در دیگر حوزه‌های سیستم‌های توصیه‌گر را نیز دارد. این رویکرد به طور خاص در شرایطی که داده‌های کاربر محدود است، مانند مشکل شروع سرد ، عملکرد بهتری از خود نشان می‌دهد.

\item
تاکید بر حریم خصوصی کاربران از طریق پیاده‌سازی حق فراموشی

پژوهش حاضر با ارائه مکانیزمی برای حذف داده‌های کاربران، به مسئله حفظ حریم خصوصی به صورت جدی پرداخته است. حق فراموشی به کاربران اجازه می‌دهد داده‌های خود را به صورت پویا حذف کنند بدون اینکه عملکرد کلی سیستم تحت تأثیر قرار گیرد. این رویکرد نه تنها اعتماد کاربر را افزایش می‌دهد، بلکه با قوانین جهانی حفظ حریم خصوصی هماهنگ است . این موضوع به عنوان یکی از نوآوری‌های پژوهش شناخته می‌شود که در ادبیات پیشین به ندرت مورد توجه قرار گرفته است.

\item
حل مشکلات شروع سرد با رویکردهای متنوع

مشکل شروع سرد یکی از چالش‌های سیستم‌های توصیه‌گر و گفتگوی وظیفه‌گرا است. این پژوهش با استفاده از ترکیبی از رویکردهای مختلف مانند فیلتر مشارکتی مبتنی بر آیتم و تحلیل احساسات، به حل این مشکل پرداخته است. به عنوان مثال، از تحلیل احساسات برای استخراج علایق کاربر از برچسب‌ها و نظرات اولیه استفاده شده است. همچنین، فیلتر مشارکتی مبتنی بر آیتم به شناسایی علایق مشترک بین کاربران جدید و قدیمی کمک کرده است. این رویکردها نه تنها به کاهش مشکل شروع سرد کمک می‌کنند، بلکه دقت و تنوع پیشنهادات برای هرکاربر را نیز افزایش می‌دهند.

\item
پیاده‌سازی و ترکیب مکانیزم‌های مختلف برای نماگر‌سازی کاربر

یکی از نوآوری‌های عملی پژوهش فوق، پیاده‌سازی و ترکیب مکانیزم‌های مختلفی مانند تحلیل احساسات و فیلتر مشارکتی مبتنی بر آیتم برای نماگر‌سازی کاربر است. با استفاده از مدل‌های از پیش آموزش‌دیده، تحلیل احساسات از برچسب‌های کاربران انجام شده و برای شناسایی ژانرها و فیلم‌های محبوب کاربر استفاده شده است. همچنین، از فیلتر مشارکتی مبتنی بر آیتم جهت یافتن آیتم‌های مورد علاقه کاربر با توجه به سابقه علاقه بقیه کاربران به آیتم‌های مشابه استفاده شده است. این ترکیب منجر به ایجاد نماگر کاربری غنی‌تر و دقیق‌تری شده است که به بهبود عملکرد سیستم کمک می‌کند.

\item
طراحی معماری چند ماژولی برای سیستم‌های گفتگوی وظیفه‌گرا

معماری پیشنهادی در این پژوهش شامل چندین ماژول است که هر یک وظایف خاصی را در سیستم گفتگوی وظیفه‌گرا انجام می‌دهند. این ماژول‌ها عبارتند از:
\begin{itemize}
\item
ماژول درک گفتگو : این ماژول برای غنی‌سازی پرس‌وجوهای کاربران طراحی شده است.
\item
ماژول تولید پاسخ : این ماژول با استفاده از مدل‌های تنظیم‌شده، پاسخ‌هایی دقیق و مرتبط تولید می‌کند.
\item
ماژول پروفایل و شخصی‌سازی : این ماژول بر اساس داده‌های نماگر کاربر و فیلترهای پویا، پاسخ‌های شخصی‌سازی‌شده‌ای ارائه می‌دهد.

\item
این معماری چند ماژولی نه تنها انعطاف‌پذیری بالایی دارد، بلکه امکان ادغام با سایر سیستم‌های هوشمند را نیز فراهم می‌کند.
\end{itemize}

\item

ارزیابی عملکرد با معیارهای متنوع

برای ارزیابی جامع سیستم، معیارهای متنوعی مانند نرخ موفقیت، نرخ تکمیل، امتیاز تعامل کاربر، دقت توصیه شخصی، و تطابق تنوع نماگر مورد استفاده قرار گرفته‌اند. این معیارها به طور خاص برای ارزیابی جنبه‌های مختلف عملکرد سیستم طراحی شده‌اند. به عنوان مثال، نرخ موفقیت و نرخ تکمیل نشان‌دهنده کارایی سیستم در انجام وظایف است، در حالی که امتیاز تعامل کاربر به سنجش رضایت کاربر می‌پردازد. این رویکرد ارزیابی چندبعدی، دقت و جامعیت نتایج را افزایش داده است.
\end{enumerate}

\subsection{تفاوت‌ها و برتری‌ها نسبت به کارهای پیشین}
پژوهش حاضر در مقایسه با کارهای پیشین، ویژگی‌ها و برتری‌های زیر را دارد:
\begin{itemize}
\item
حل مشکلات حریم خصوصی : در مقایسه با دیگر تحقیقات، این پژوهش به مسئله حفظ حریم خصوصی پرداخته و مکانیزم حذف داده‌ها را پیاده‌سازی کرده است.
\item
شخصی‌سازی پیشرفته : ترکیب سه رویکرد (نماگر‌سازی ساده، تحلیل احساسات، و فیلتر مشارکتی) باعث شده است که نماگر‌های کاربری دقیق‌تر و توصیه‌ها مرتبط‌تر شوند .
\item
کاهش نیاز به سخت‌افزار پیشرفته : با استفاده از تنظیم سریع به جای تنظیم دقیق، این پژوهش هزینه محاسباتی را کاهش داده و کارایی سیستم را بهبود بخشیده است.
\item
حل مشکل شروع سرد : راهکارهای ارائه‌شده، امکان تعامل موثر با کاربران جدید حتی در صورت عدم وجود داده‌های کافی را فراهم کرده‌است.
\end{itemize}


\section{ساختار پایان نامه}

فصل دوم بررسی ادبیات پژوهش است که شامل مروری بر اهمیت  سیستم‌های گفتگوی وظیفه‌گرا و زمینه های مرتبط آن است.

مفاهیم اصلی شامل سیستم‌های گفتگوی وظیفه‌گرا و انواع فرعی آن (به عنوان مثال، سیستم های وظیفه محور و دامنه باز)، تنظیم سریع و مزایای آن نسبت به تنظیم دقیق، تکنیک‌های نماگر کاربری، از جمله تجزیه و تحلیل احساسات و فیلتر‌کردن مشارکتی و در نهایت چالش‌هایی مانند مسائل مربوط به شروع سرد و نگرانی‌های مربوط به حریم خصوصی مورد بررسی قرار خواهد‌گرفت.

در ادامه به بررسی آثار مرتبط مقایسه تفصیلی تحقیقات قبلی با تمرکز بر اثربخشی تنظیم سریع در پردازش زبان طبیعی, نقش شخصی‌سازی در  سیستم‌های گفتگوی وظیفه‌گرا و محدودیت‌های رویکردهای موجود در راه‌حل‌های حریم خصوصی و شروع سرد پرداخته شده‌است.

این بخش با شناسایی شکاف‌هایی به پایان می‌رسد که چارچوب پیشنهادی به دنبال رفع آن است.

فصل سوم روش‌شناسی است که در ابتدا رویکرد کلی و طرح تحقیق را توضیح می‌دهد.

جمع‌آوری و آماده‌سازی داده‌ها و جزئیات استفاده از مجموعه داده‌های مووی‌لنز  و تبدیل آن به داده‌های مکالمه مانند به تفضیل شرح داده شده است.

در ادامه معماری سیستم و توضیح به تفیکی برای هر ماژول (به عنوان مثال، درک گفتگو، منطق شخصی سازی)، جریان پرسه از زمان ورود به سیستم و  پردازش آن در معماری فوق آورده شده است.

پیاده‌سازی‌های خاص ماژول مانند درک گفتگو استخراج موجودیت و اصلاح پرس و جو، ماژول تولید پاسخ با بهره‌گیری تنظیم سریع و انطباق با حق فراموشی و همچنین منطق شخصی‌سازی یعنی روش‌های نماگر‌سازی کاربر، تجزیه و تحلیل احساسات، و فیلتر‌کردن مشارکتی ارایه شده است.


فصل چهارم شامل نتایج ارزیابی نهایی سیستم است.

در ابتدا به بررسی اجمالی مجموعه‌داده مووی‌لنز و پیش پردازش آن پرداخته شده است.

نتایج ارزیابی برای معیارهای مختلف ارایه شده و  تجزیه‌و‌تحلیل یافته‌ها به همراه آنالیز حساسیت هر بخش سیستم مانند تاثیر نماگر کاربری بر کیفیت پاسخ  نقش مکانیسم های حق فراموشی در اعتماد کاربر و عملکرد سیستم شرح داده‌شده است.


فصل پنجم بحث و نتیجه گیری است که اهداف و روش تحقیق را بیان می‌کند.

یافته های کلیدی مانند بهبود عملکرد  سیستم‌های گفتگوی وظیفه‌گرا با تنظیم سریع و نماگر کاربری، افزایش تعامل کاربر از طریق توصیه‌های شخصی‌شده، استفاده از حق‌فراموشی و پرداختن به مسائل مربوط به شروع سرد و حریم خصوصی شرح داده شده است.

در ادامه به محدودیت‌ها و چالش‌های برخورد شده در مسیر پژوهش مانند محدودیت‌های سخت‌افزاری که اندازه مدل و قابلیت‌های تنظیم دقیق را محدود می‌کند و چالش‌های ایجاد مجموعه داده و مقیاس‌پذیری پرداخته شده است.

همچنین در مورد کار آینده و کاوش در روش‌های پیشرفته نماگر و مجموعه داده‌های غنی‌تر به همراه گسترش چارچوب به حوزه‌های دیگر فراتر از توصیه‌های فیلم صحبت شده است.
