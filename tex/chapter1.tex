% !TeX root=../main.tex

\chapter{پیش گفتار}

\section{مقدمه}

سیستم‌های گفتگوی وظیفه‌گرا به عنوان یکی از مهم‌ترین ابزارهای تعامل انسان و ماشین، نقش اساسی در بهبود تجربه کاربری در حوزه‌هایی مانند سیستم‌های توصیه‌گر، خدمات مشتری و تجارت الکترونیک ایفا می‌کنند. این سیستم‌ها با درک پرسش‌های کاربران و ارائه پاسخ‌های متناسب، تجربه‌ای طبیعی‌تر و مؤثر را در تعاملات کاربران با سیستم فراهم می‌کنند. با این حال، چالش‌های زیادی از جمله نیاز به تنظیم دقیق مدل‌های زبانی، شخصی‌سازی پاسخ‌ها بر اساس ترجیحات کاربران، مشکل شروع سرد برای کاربران جدید و نگرانی‌های مربوط به حریم خصوصی همچنان به عنوان موانع اصلی در مسیر پیشرفت این فناوری‌ها مطرح هستند.  

یکی از مهم‌ترین چالش‌های این حوزه، شخصی‌سازی پاسخ‌ها و انطباق‌پذیری سیستم با نیازهای خاص هر کاربر است. در بسیاری از روش‌های موجود، اطلاعات کاربر به صورت ایستا در سیستم ذخیره می‌شود، اما عدم تطبیق آن با تغییرات در رفتار و سلیقه کاربران باعث کاهش دقت توصیه‌ها و کاهش رضایت کاربر به مرور زمان می‌شود. علاوه بر این، مشکل شروع سرد، یعنی ناتوانی سیستم در ارائه پاسخ‌های دقیق به کاربران جدید که سابقه‌ای در سیستم ندارند، یکی از مسائل جدی در سیستم‌های توصیه‌گر و گفتگوی وظیفه‌گرا است. از سوی دیگر، حفظ حریم خصوصی کاربران یکی دیگر از چالش‌های کلیدی در این زمینه است؛ کاربران نیاز دارند که داده‌هایشان در صورت درخواست حذف شود، اما بسیاری از روش‌های سنتی قابلیت اجرایی مؤثری در این زمینه ندارند.  

در کنار این چالش‌ها، روش‌های تنظیم دقیق مدل‌های زبانی، اگرچه باعث بهبود عملکرد سیستم‌های گفتگوی وظیفه‌گرا می‌شوند، اما به دلیل نیاز به منابع محاسباتی بسیار بالا و زمان طولانی پردازش، محدودیت‌هایی را ایجاد می‌کنند. علاوه بر این، روش‌های سنتی تنظیم دقیق معمولاً نیازمند حجم وسیعی از داده‌های آموزش هستند که در بسیاری از موارد به‌راحتی در دسترس نیستند. به همین دلیل، در سال‌های اخیر استفاده از روش تنظیم سریع مورد توجه قرار گرفته است، چرا که این روش بدون نیاز به تغییر تمام پارامترهای مدل، امکان انطباق سریع‌تر و مؤثرتر را با دامنه‌های خاص فراهم می‌کند.  

این پژوهش با هدف ارائه یک چارچوب جدید برای بهبود سیستم‌های گفتگوی وظیفه‌گرا، ترکیبی از تکنیک‌های تنظیم سریع و پروفایل‌سازی پویا را پیشنهاد می‌دهد. در این روش، ابتدا داده‌های مرتبط با کاربر از طریق ترکیب حلیل احساسات برچسب‌های کاربر، فیلتر مشارکتی مبتنی بر آیتم، و تحلیل بازخورد کاربر استخراج شده و سپس با استفاده از تکنیک‌های تنظیم سریع و با استفاده از قدرت مدل های زبانی برای پاسخ‌گویی دقیق‌تر و شخصی‌سازی‌شده تنظیم می‌شود. این سیستم همچنین امکان حذف داده‌های کاربر را مطابق با حق فراموشی فراهم می‌کند تا از حریم خصوصی کاربران محافظت شود.  

با توجه به این رویکرد نوآورانه، انتظار می‌رود که نتایج این تحقیق بتواند راه را برای پیشرفت‌های آینده در حوزه شخصی‌سازی سیستم‌های گفتگوی وظیفه‌گرا هموار کند و الگویی برای بهبود تعاملات کاربر-ماشین در حوزه‌های مختلف از جمله تجارت الکترونیک، خدمات مشتریان، و سیستم‌های یادگیری هوشمند فراهم آورد.

\section{بیان مسئله}
تحقیقات موجود در حوزه سیستم‌های گفتگوی وظیفه‌گرا به رغم پیشرفت‌های چشمگیر، همچنان دارای کاستی‌هایی هستند:
\begin{itemize}
\item
نبود تمرکز بر حفظ حریم خصوصی کاربران: بسیاری از سیستم‌های موجود داده‌های کاربر را بدون امکان حذف یا اصلاح ذخیره می‌کنند، که می‌تواند منجر به نگرانی‌های امنیتی و اخلاقی شود.
\item
پروفایل‌سازی ناکارآمد: روش‌های موجود در پروفایل‌سازی کاربران اغلب تنها به استفاده از یک تکنیک محدود هستند و از روش هایی مانند بازخورد موثر کاربران و پروفایل پویا برای به‌روزرسانی مدل‌ها استفاده نمی‌کنند.
\item
نیاز به منابع سخت‌افزاری قدرتمند: تنظیم دقیق مدل‌های زبانی به منابع عظیم سخت‌افزاری نیاز دارد که اجرای آن را برای بسیاری از کاربران و سازمان‌ها دشوار می‌کند.
\end{itemize}

روش‌های متعددی با استفاده از متدهای مختلف هم با و هم بدون استفاده از مدل‌های زبانی طی سال ‌های اخیر ارایه شده است اما آنها نیز دارای کاستی‌هایی در کار خود هستند:
\begin{itemize}
\item
نبود تنوع در روش‌های پروفایل‌سازی: تحقیقات فعلی عمدتاً به یک رویکرد محدود می‌شوند، در حالی که ترکیب روش‌های مختلف می‌تواند به دقت بیشتری در شخصی‌سازی پاسخ‌ها منجر شود که در نهایت به بهبود سیستم و افزایش کیفیت مکالمه و رضایت کاربر منجر می شوند.
\item
مشکلات شروع سرد: سیستم‌های موجود در مواجهه با کاربران جدید یا داده‌های ناکافی، عملکرد ضعیفی دارند.
\item
عدم امکان تنظیم سریع: بیشتر تحقیقات به تنظیم دقیق مدل وابسته هستند که نه تنها زمان‌بر و پرهزینه است، بلکه انعطاف‌پذیری کمتری نسبت به تنظیم سریع دارد.
\item
عدم رعایت حق فراموشی: کاربران نمی‌توانند داده‌های خود را حذف یا به‌روزرسانی کنند، که این مسئله اعتماد به سیستم را کاهش می‌دهد.
\end{itemize}


\section{اهداف پژوهش}

اهداف پژوهش باید منعکس‌کننده چالش‌های اصلی، گپ‌های موجود در ادبیات، و راه‌حل‌هایی باشد که تحقیق حاضر ارائه می‌دهد. این اهداف در راستای دستیابی به یک سیستم گفتگوی وظیفه‌گرا با قابلیت‌های شخصی‌سازی پیشرفته، تنظیم سریع، و رعایت حریم خصوصی طراحی شده‌اند. در این بخش، اهداف پژوهش به دو دسته اصلی تقسیم می‌شوند: اهداف کلی و اهداف جزئی.

\begin{itemize}
\item
اهداف کلی
\begin{enumerate}
\item
طراحی و توسعه یک سیستم گفتگوی وظیفه‌گرا با قابلیت تنظیم سریع

این هدف شامل ارائه روشی است که مدل‌های زبانی بزرگ را به صورت کارآمد و مقرون‌به‌صرفه از نظر هزینه محاسباتی و زمانی تنظیم کند و به نتایج بهینه در تولید پاسخ‌های مرتبط دست یابد.
\item

افزایش سطح شخصی‌سازی در سیستم‌های گفتگوی وظیفه‌گرا

روش پیشنهادی باید قادر به شناسایی و تطبیق با علایق و ترجیحات کاربران، حتی در مواجهه با کاربران جدید (مشکل شروع سرد) باشد.
\end{enumerate}

\item
اهداف جزئی

\begin{enumerate}
\item
توسعه تکنیک‌های پیشرفته پروفایل‌سازی کاربر

استفاده از ترکیب سه روش متفاوت در پروفایل‌سازی کاربر که منجر به ایجاد پروفایلی متنوع خواهدشد که در طول زمان نیز پویایی خودرا حفظ کرده و به روز می‌شود.

شناسایی ژانرهای برتر و فیلم‌های موردعلاقه کاربران با استفاده از داده‌های تاریخی، تحلیل احساسات بر اساس برچسب‌های کاربران برای استخراج فیلم‌های مرتبط با اولویت‌های احساسی و همچنین استفاده از فیلتر مشارکتی مبتنی بر آیتم برای پیشنهاد فیلم‌های جدید به کاربران از جمله این روش‌ها هستند.

\item
کاهش مشکلات سخت‌افزاری و منابع مورد نیاز

جایگزینی تنظیم دقیق با تنظیم سریع برای کاهش بار پردازشی و نیاز به سخت‌افزارهای قدرتمند باعث کاهش هزینه‌های محاسباتی و همچنین زمانی در مدل شد.

\item
طراحی یک چارچوب انعطاف‌پذیر برای انطباق با داده‌های جدید

ارائه سیستمی که بتواند به‌طور مداوم پروفایل کاربران را به‌روزرسانی کرده و با داده‌های جدید سازگار شود و همچنین امکان استفاده از یادگیری چند شات برای بهبود دقت پاسخ‌ها در مواجهه با ورودی‌های جدید از جمله موارد مورد توجه در پیاده‌سازی مدل است.


\item
تعریف متریک‌های ارزیابی جدید

معیار‌هایی برای ارزیابی کیفیت نتایج و همچنین متناسب با زمینه خاص تحقیق معرفی و استفاده شدند. علاوه بر آن معیارهایی که مختص ارزیابی پروفایل کاربر مانند محاسبه امتیاز درگیری کاربر به‌عنوان شاخصی برای سنجش رضایت و تعامل کاربر نیز مورد استفاده قرار گرفتند.

\item
حل مشکلات شروع سرد

با استفاده از ژانرهای هدف و داده‌های مشارکتی برای ارائه توصیه‌های معنادار به کاربران جدید به مشکل شروع سرع این دسته کاربران پرداخته شد.

\item
تضمین امنیت و حریم خصوصی داده‌ها

مکانیزمی برای ناشناس‌سازی داده‌ها و امکان حذف اطلاعات در صورت درخواست کاربر جهت حفظ هرچه بیشتر حریم خصوصی کاربران و همچنین افزایش اعتماد آنها به استفاده از سیستم ارائه گردید.

\end{enumerate}

\end{itemize}

اهداف پژوهش فوق از آن جهت دارای اهمیت است که ترکیب پروفایل‌سازی چندگانه، تنظیم سریع، و رعایت حریم خصوصی باعث افزایش اثربخشی سیستم‌های توصیه‌گر می‌شود. همچنین این سیستم قابلیت استفاده در حوزه‌های مختلف مانند خدمات مشتری، آموزش، و تجارت الکترونیک را داراست. ارائه پاسخ‌های شخصی‌سازی‌شده و سریع باعث افزایش رضایت کاربران و در نهایت افزایش تعامل آن‌ها با سیستم می‌شود.


\section{فرضیه‌های پژوهش}


فرضیه اصلی پژوهش به صورت زیر است.

استفاده از روش ترکیبی پروفایل‌سازی کاربر به همراه تنظیم سریع در سیستم‌های گفتگوی وظیفه‌گرا می‌تواند به طور موثری منجر به ارتقای عملکرد آن‌ها در پاسخگویی به نیازهای کاربران گردد.

این فرضیه به صورت مستقیم به بررسی نحوه استفاده از پروفایل کاربر و تنظیم سریع برای بهبود معیارهایی چون رضایت کاربر، تعامل، و دقت پاسخ‌ها می‌پردازد. ترکیب این دو روش، با رویکردی جامع‌تر، مشکلاتی نظیر شروع سرد، حریم خصوصی و تعاملات بلندمدت کاربر را هدف قرار می‌دهد.


\section{سوال‌های پژوهش}

در این بخش، سوالات پژوهش به صورت جامع و دقیق با توجه به اهداف و یافته‌های تحقیق بیان شده است. این فرضیات مبتنی بر شکاف‌های موجود در ادبیات پژوهشی و نیازهای سیستم‌های گفتگوی وظیفه‌گرا تدوین شده‌اند.

\begin{enumerate}
\item
تنظیم سریع و تاثیر آن بر سیستم‌های گفتگوی وظیفه‌گرا

استفاده از تنظیم سریع می‌تواند با کاهش نیاز به منابع پردازشی و زمان آموزش، منجر به بهبود عملکرد سیستم‌های گفتگوی وظیفه‌گرا شود.

تنظیم سریع به جای تنظیم دقیق، امکان بهره‌برداری از مدل‌های زبانی بزرگ را برای وظایف خاص با داده‌های محدود را فراهم می‌کند. این روش منجر به کاهش مصرف منابع سخت‌افزاری، افزایش سرعت پاسخ‌دهی و بهبود کیفیت پاسخ‌ها می‌شود. مزیت دیگر این روش، سازگاری آن با داده‌های جدید است که سیستم را در مواجهه با ورودی‌های متنوع تقویت می‌کند.


\item
تاثیر استفاده از داده‌های پروفایل کاربر بر تعامل و رفتار کاربران

استفاده از داده‌های پروفایل کاربر می‌تواند تعاملات کاربران با سیستم گفتگوی وظیفه‌گرا را بهبود بخشد و منجر به افزایش دقت و شخصی‌سازی پاسخ‌ها شود.

تحلیل داده‌های کاربران مانند ژانرهای محبوب، فیلم‌های مورد علاقه و بازخوردها به سیستم امکان می‌دهد تا پاسخ‌های متناسب با ترجیحات شخصی هر کاربر تولید کند. استفاده از تکنیک‌های پروفایل‌سازی پیشرفته مانند تحلیل احساسات و فیلتر مشارکتی، تجربه کاربری را بهبود می‌بخشد. این سوال بررسی می‌کند که چگونه تعاملات طولانی‌مدت کاربران با سیستم منجر به بهبود پروفایل‌ها و افزایش رضایت کاربران می‌شود.

\item
ویژگی‌ها و شاخص‌های کلیدی در پروفایل‌سازی کاربران

شناسایی و استفاده از ویژگی‌ها و شاخص‌های کلیدی در پروفایل‌سازی کاربران می‌تواند عملکرد کلی سیستم‌های گفتگوی وظیفه‌گرا را بهبود بخشد.

شاخص‌های کلیدی شامل ژانرهای مورد علاقه، تحلیل احساسات و بازخورد کاربران می‌باشد. این سوال بررسی می‌کند که چگونه ترکیب روش‌های مختلف پروفایل‌سازی، مانند تحلیل احساسی و داده‌های گذشته کابران در سیستم، دقت سیستم را افزایش می‌دهد. همچنین ارزیابی می‌شود که چه مکانیزم‌هایی برای حفظ حریم خصوصی (مانند حق فراموشی) باید در سیستم‌های پروفایل‌سازی گنجانده شوند.


\item
رفع مشکلات شروع سرد در سیستم‌های توصیه‌گر و گفتگوی وظیفه‌گرا

ترکیب روش‌های فیلتر مشارکتی مبتنی بر آیتم و استفاده از داده‌های ژانر هدف می‌تواند مشکل شروع سرد را در سیستم‌های گفتگوی وظیفه‌گرا به طور موثری کاهش دهد.

کاربران جدید معمولاً فاقد داده‌های کافی برای ارائه توصیه‌های معنادار هستند. این سوال بررسی می‌کند که چگونه استفاده از روش‌های جایگزین، مانند درخواست مستقیم ژانرهای هدف از کاربر، مشکل شروع سرد را حل می‌کند.همچنین ارزیابی می‌شود که ترکیب این داده‌ها با تنظیم سریع چگونه می‌تواند بهبود قابل توجهی ایجاد کند.

\end{enumerate}


\section{نوآوری‌ها و سهم پژوهش}

در این بخش، نوآوری‌ها و سهم علمی پژوهش حاضر با توجه به دستاوردها و ایده‌های منحصربه‌فرد ارائه می‌شود. این نوآوری‌ها در دو بعد اصلی تئوریک و عملی دسته‌بندی شده‌اند.


نوآوری‌های تئوریک شامل موارد زیر است.
\begin{enumerate}
\item
معرفی رویکرد ترکیبی تنظیم سریع و پروفایل‌سازی کاربر

این پژوهش برای اولین بار استفاده از تنظیم سریع و پروفایل‌سازی پیشرفته کاربر را در سیستم‌های گفتگوی وظیفه‌گرا ترکیب کرده است. این روش نشان می‌دهد که چگونه ترکیب داده‌های شخصی‌سازی‌شده و تنظیم سریع می‌تواند به بهبود پاسخ‌دهی و رضایت کاربر منجر شود. همچنین یک مدل مفهومی برای ادغام داده‌های کاربر و تنظیم سریع ارائه می‌کند که قابلیت استفاده در دیگر حوزه‌های سیستم‌های توصیه‌گر را نیز دارد.
\item
تاکید بر حریم خصوصی کاربران از طریق پیاده‌سازی حق فراموشی

این پژوهش با ارائه مکانیزمی برای حذف داده‌های کاربران، یک رویکرد تئوریک برای تعامل بین حفظ حریم خصوصی و عملکرد سیستم پیشنهاد می‌کند.

\item
بررسی عمیق مشکلات شروع سرد

این پژوهش رویکردهای متنوعی مانند فیلتر مشارکتی و جمع‌آوری داده‌های اولیه از کاربران را برای حل مشکل شروع سرد در سیستم‌های توصیه‌گر تحلیل و عملیاتی کرده است.
\end{enumerate}



نوآوری‌های عملی نیز شامل موارد زیر است.
\begin{enumerate}
\item
پیاده‌سازی و ترکیب مکانیزم‌های مختلف مانند تحلیل احساسات و فیلترمشارکتی مبتنی بر آیتم برای پروفایل‌سازی کاربر

با استفاده از مدل‌های از پیش آموزش‌دیده، تحلیل احساسات از برچسب‌های کاربران انجام شده و برای شناسایی ژانرها و فیلم‌های محبوب کاربر استفاده شده است. همچنین از فیلترمشارکتی مبتنی بر آیتم جهت یافتن آیتم‌های مورد علاقه کاربر با توجه به سابقه علاقه بقیه کاربران به آیتم‌های مشابه استفاده شده است تا پروفایل کاربری غنی‌تر و بهتری ایجاد گردد.

\item
طراحی معماری چند ماژولی برای سیستم‌های گفتگوی وظیفه‌گرا

این معماری شامل ماژول درک گفتگو برای غنی‌سازی پرس‌وجوهای کاربران، ماژول تولید پاسخ با استفاده از مدل‌های تنظیم‌شده و همچنین ماژول پرفایل و شخصی‌سازی بر اساس داده‌های پروفایل کاربر و فیلترهای پویا است.

\item
ارزیابی عملکرد مدل با معیارهای متنوع

معیارهایی مانند نرخ موفقیت، نرخ تکمیل، و امتیاز تعامل کاربر و همچنین معیارهای خاص پروفایل کاربری مانند دقت توصیه شخصی و تطابق تنوع نمایه برای ارزیابی جامع سیستم معرفی و استفاده شدند.
\end{enumerate}



تفاوت‌ها و برتری‌ها نسبت به کارهای پیشین به شرح زیر است.
\begin{itemize}
\item
حل مشکلات حریم خصوصی

در مقایسه با دیگر تحقیقات، این پژوهش به مسئله حفظ حریم خصوصی پرداخته و مکانیزم حذف داده‌ها را پیاده‌سازی کرده است.
\item
شخصی‌سازی پیشرفته

ترکیب سه رویکرد (پروفایل‌سازی ساده، تحلیل احساسات، و فیلتر مشارکتی) باعث شده است که پروفایل‌های کاربری دقیق‌تر و توصیه‌ها مرتبط‌تر شوند.
\item
کاهش نیاز به سخت‌افزار پیشرفته

با استفاده از تنظیم سریع به جای تنظیم دقیق، این پژوهش هزینه محاسباتی را کاهش داده و کارایی سیستم را بهبود بخشیده است.
\item
حل مشکل شروع سرد

راهکارهای ارائه‌شده، امکان تعامل موثر با کاربران جدید حتی در صورت عدم وجود داده‌های کافی را فراهم کرده‌است.
\end{itemize}

این پژوهش با ارائه یک رویکرد جدید ترکیبی و تاکید بر حریم خصوصی و شخصی‌سازی، سهم مهمی در پیشرفت سیستم‌های گفتگوی وظیفه‌گرا داشته و می‌تواند به عنوان الگویی برای سایر سیستم‌های توصیه‌گر استفاده شود.



\section{ساختار پایان نامه}

فصل دوم بررسی ادبیات پژوهش است که شامل مروری بر اهمیت  سیستم‌های گفتگوی وظیفه‌محور و زمینه های مرتبط آن است.

مفاهیم اصلی شامل سیستم‌های گفتگوی وظیفه‌محور و انواع فرعی آن (به عنوان مثال، سیستم های وظیفه محور و دامنه باز)، تنظیم سریع و مزایای آن نسبت به تنظیم دقیق، تکنیک‌های پروفایل کاربری، از جمله تجزیه و تحلیل احساسات و فیلتر‌کردن مشارکتی و در نهایت چالش‌هایی مانند مسائل مربوط به شروع سرد و نگرانی‌های مربوط به حریم خصوصی مورد بررسی قرار خواهد‌گرفت.

در ادامه به بررسی آثار مرتبط مقایسه تفصیلی تحقیقات قبلی با تمرکز بر اثربخشی تنظیم سریع در پردازش زبان طبیعی, نقش شخصی‌سازی در  سیستم‌های گفتگوی وظیفه‌محور و محدودیت‌های رویکردهای موجود در راه‌حل‌های حریم خصوصی و شروع سرد پرداخته شده‌است.

این بخش با شناسایی شکاف‌هایی به پایان می‌رسد که چارچوب پیشنهادی به دنبال رفع آن است.

فصل سوم روش‌شناسی است که در ابتدا رویکرد کلی و طرح تحقیق را توضیح می‌دهد.

جمع‌آوری و آماده‌سازی داده‌ها و جزئیات استفاده از مجموعه داده‌های مووی‌لنز  و تبدیل آن به داده‌های مکالمه مانند به تفضیل شرح داده شده است.

در ادامه معماری سیستم و توضیح به تفیکی برای هر ماژول (به عنوان مثال، درک گفتگو، منطق شخصی سازی)، جریان پرسه از زمان ورود به سیستم و  پردازش آن در معماری فوق آورده شده است.

پیاده‌سازی‌های خاص ماژول مانند درک گفتگو استخراج موجودیت و اصلاح پرس و جو، ماژول تولید پاسخ با بهره‌گیری تنظیم سریع و انطباق با حق فراموشی و همچنین منطق شخصی‌سازی یعنی روش‌های پروفایل‌سازی کاربر، تجزیه و تحلیل احساسات، و فیلتر‌کردن مشارکتی ارایه شده است.


فصل چهارم شامل نتایج ارزیابی نهایی سیستم است.

در ابتدا به بررسی اجمالی مجموعه‌داده مووی‌لنز و پیش پردازش آن پرداخته شده است.

نتایج ارزیابی برای معیارهای مختلف ارایه شده و  تجزیه‌و‌تحلیل یافته‌ها به همراه آنالیز حساسیت هر بخش سیستم مانند تاثیر پروفایل کاربری بر کیفیت پاسخ  نقش مکانیسم های حق فراموشی در اعتماد کاربر و عملکرد سیستم شرح داده‌شده است.


فصل پنجم بحث و نتیجه گیری است که اهداف و روش تحقیق را بیان می‌کند.

یافته های کلیدی مانند بهبود عملکرد  سیستم‌های گفتگوی وظیفه‌محور با تنظیم سریع و پروفایل کاربری، افزایش تعامل کاربر از طریق توصیه‌های شخصی‌شده، استفاده از حق‌فراموشی و پرداختن به مسائل مربوط به شروع سرد و حریم خصوصی شرح داده شده است.

در ادامه به محدودیت‌ها و چالش‌های برخورد شده در مسیر پژوهش مانند محدودیت‌های سخت‌افزاری که اندازه مدل و قابلیت‌های تنظیم دقیق را محدود می‌کند و چالش‌های ایجاد مجموعه داده و مقیاس‌پذیری پرداخته شده است.

همچنین در مورد کار آینده و کاوش در روش‌های پیشرفته پروفایل و مجموعه داده‌های غنی‌تر به همراه گسترش چارچوب به حوزه‌های دیگر فراتر از توصیه‌های فیلم صحبت شده است.
