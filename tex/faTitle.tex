% !TeX root=../main.tex
% در این فایل، عنوان پایان‌نامه، مشخصات خود، متن تقدیمی‌، ستایش، سپاس‌گزاری و چکیده پایان‌نامه را به فارسی، وارد کنید.
% توجه داشته باشید که جدول حاوی مشخصات پروژه/پایان‌نامه/رساله و همچنین، مشخصات داخل آن، به طور خودکار، درج می‌شود.
%%%%%%%%%%%%%%%%%%%%%%%%%%%%%%%%%%%%
% دانشگاه خود را وارد کنید
\university{دانشگاه اصفهان}
% دانشکده، آموزشکده و یا پژوهشکده  خود را وارد کنید
\faculty{دانشکده مهندسی کامپیوتر}
% گروه آموزشی خود را وارد کنید (در صورت نیاز)
\department{گروه مهندسی نرم افزار}
% رشته تحصیلی خود را وارد کنید
\subject{مهندسی نرم افزار}
% گرایش خود را وارد کنید

% عنوان پایان‌نامه را وارد کنید
\title{موضوع پایان نامه شما}
% نام استاد راهنما را وارد کنید
\firstsupervisor{دکتر راهنمای اول}
\firstsupervisorrank{استاد}

% نام داور خود را وارد نمایید.
\internaljudge{دکتر داور داخلی}
\internaljudgerank{دانشیار}

% نام دانشجو را وارد کنید
\name{مهرداد}
% نام خانوادگی دانشجو را وارد کنید
\surname{قصابی}
% شماره دانشجویی دانشجو را وارد کنید
\studentID{973613060}
% تاریخ پایان‌نامه را وارد کنید
\thesisdate{شهریور ۱۴۰۱}
% به صورت پیش‌فرض برای پایان‌نامه‌های کارشناسی تا دکترا به ترتیب از عبارات «پروژه»، «پایان‌نامه» و «رساله» استفاده می‌شود؛ اگر  نمی‌پسندید هر عنوانی را که مایلید در دستور زیر قرار داده و آنرا از حالت توضیح خارج کنید.
%\projectLabel{پایان‌نامه}

% به صورت پیش‌فرض برای عناوین مقاطع تحصیلی کارشناسی تا دکترا به ترتیب از عبارت «کارشناسی»، «کارشناسی ارشد» و «دکتری» استفاده می‌شود؛ اگر نمی‌پسندید هر عنوانی را که مایلید در دستور زیر قرار داده و آنرا از حالت توضیح خارج کنید.
%\degree{}
%%%%%%%%%%%%%%%%%%%%%%%%%%%%%%%%%%%%%%%%%%%%%%%%%%%%
%% پایان‌نامه خود را تقدیم کنید! %%
\dedication
{
{\Large تقدیم به:}\\
\begin{flushleft}{
	\huge
	مادرم که همه درد هایم را مرهم است\\
}
\end{flushleft}
}
%% متن قدردانی %%
%% ترجیحا با توجه به ذوق و سلیقه خود متن قدردانی را تغییر دهید.
\acknowledgement{
سپاس خداوندگار حکیم را که با لطف بی‌کران خود، آدمی را به زیور عقل آراست.

در آغاز وظیفه‌  خود  می‌دانم از زحمات بی‌دریغ اساتید  راهنمای خود،  جناب آقای دکتر ... و ...، صمیمانه تشکر و  قدردانی کنم که در طول انجام این پایان‌نامه با نهایت صبوری همواره راهنما و مشوق من بودند و قطعاً بدون راهنمایی‌های ارزنده‌ ایشان، این مجموعه به انجام نمی‌رسید.

از جناب آقای دکتر ... که  زحمت مشاوره‌، بازبینی و تصحیح این پایان‌نامه را تقبل فرمودند کمال امتنان را دارم.


و در پایان، بوسه می‌زنم بر دستان خداوندگاران مهر و مهربانی، پدر و مادر عزیزم و بعد از خدا، ستایش می‌کنم وجود مقدس‌شان را و تشکر می‌کنم از خانواده عزیزم به پاس عاطفه سرشار و گرمای امیدبخش وجودشان، که بهترین پشتیبان من بودند.
}
%%%%%%%%%%%%%%%%%%%%%%%%%%%%%%%%%%%%
%چکیده پایان‌نامه را وارد کنید
\fa-abstract{
این راهنما، نمونه‌ای از قالبِ پروژه، پایان‌نامه و رسالهٔ دانشگاه اصفهان می‌باشد که با استفاده از کلاس 
\lr{tehran-thesis}
و بستهٔ زی‌پرشین در \lr{\LaTeX}{} تهیه شده است. این قالب به گونه‌ای طراحی شده است که مطابق با دستورالعمل نگارش و تدوین پایان‌نامه کارشناسی ارشد و دکتری، مورخ ۹۳/۰۶/۰۳ پردیس دانشکده‌های فنی دانشگاه اصفهان باشد و حروف‌چینی بسیاری از قسمت‌های آن، مطابق با استاندارد قالب‌های فارسی پایان‌نامه در لاتک، به طور خودکار انجام می‌شود.

چکیده بخشی از پایان‌نامه است که خواننده را به مطالعه آن علاقمند می‌کند و یا از آن می‌گریزاند. چکیده باید ترجیحاً‌ در یک صفحه باشد. در نگارش چکیده نکات زیر باید رعایت شود. متن چکیده باید مزین به کلمه‌ها و عبارات سلیس، آشنا، بامعنی و روشن باشد. بگونه‌ای که با حدود ۳۰۰ تا ۵۰۰ کلمه بتواند خواننده را به خواندن پایان‌نامه راغب نماید. چکیده، جدای از پایان‌نامه باید به تنهایی گویا و مستقل باشد. در چکیده باید از ذکر منابع، اشاره به جداول و نمودارها اجتناب شود.تمیز بودن مطلب، نداشتن غلط‌های املایی یا دستور زبانی و رعایت دقت و تسلسل روند نگارش چکیده از نکات مهم دیگری است که باید درنظر گرفته شود. در چکیده پایان‌نامه باید از درج مشخصات مربوط به پایان‌نامه خودداری شود.
چکیده باید منعکس‌کننده اصل موضوع باشد. در چکیده باید اهداف تحقیق مورد توجه قرار گیرد. تأکید روی اطلاعات تازه (یافته‌ها) و اصطلاحات جدید یا نظریه‌ها، فرضیه‌ها، نتایج و پیشنهادها متمرکز شود. اگر در پایان‌نامه روش نوینی برای اولین بار ارائه می‌شود و تا به حال معمول نبوده است، با جزئیات بیشتری ذکر شود. شایان ذکر است چکیده فارسی و انگلیسی باید حتماً به تأیید استاد راهنما رسیده باشد.

کلمات کلیدی در انتهای چکیده فارسی و انگلیسی آورده می‌شود. محتوای چکیده‌ها بر اساس موضوع و گرایش تحقیق طبقه‌بندی می‌شود و به همین جهت وجود کلمات شاخص و کلیدی، مراکز اطلاعاتی  را در طبقه‌بندی دقیق و سریع پایان‌نامه یاری می‌دهد. کلمات کلیدی، راهنمای نکات مهم موجود در پایان‌نامه هستند. بنابراین باید در حد امکان کلمه‌ها یا عباراتی انتخاب شود که ماهیت، محتوا و گرایش کار را به وضوح روشن نماید.
}
% کلمات کلیدی پایان‌نامه را وارد کنید
\keywords{حداکثر ۵ کلمه یا عبارت، متناسب با عنوان، قالب پایان‌نامه، لاتک}
% انتهای وارد کردن فیلد‌ها
%%%%%%%%%%%%%%%%%%%%%%%%%%%%%%%%%%%%%%%%%%%%%%%%%%%%%%
