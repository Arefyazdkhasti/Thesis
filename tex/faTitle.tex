% !TeX root=../main.tex
% در این فایل، عنوان پایان‌نامه، مشخصات خود، متن تقدیمی‌، ستایش، سپاس‌گزاری و چکیده پایان‌نامه را به فارسی، وارد کنید.
% توجه داشته باشید که جدول حاوی مشخصات پروژه/پایان‌نامه/رساله و همچنین، مشخصات داخل آن، به طور خودکار، درج می‌شود.
%%%%%%%%%%%%%%%%%%%%%%%%%%%%%%%%%%%%
% دانشگاه خود را وارد کنید
\university{دانشگاه اصفهان}
% دانشکده، آموزشکده و یا پژوهشکده  خود را وارد کنید
\faculty{دانشکده مهندسی کامپیوتر}
% گروه آموزشی خود را وارد کنید (در صورت نیاز)
\department{گروه مهندسی نرم افزار}
% رشته تحصیلی خود را وارد کنید
\subject{مهندسی نرم افزار}
% گرایش خود را وارد کنید

% عنوان پایان‌نامه را وارد کنید
\title{ارتقای سيستم گفتگوی وظيفه گرا از طریق تنظيم سریع و پروفایل سازی کاربر}
% نام استاد راهنما را وارد کنید
\firstsupervisor{دکتر افسانه فاطمی}
\firstsupervisorrank{استاد}

% نام داور خود را وارد نمایید.
\internaljudge{دکتر داور داخلی}
\internaljudgerank{دانشیار}

% نام دانشجو را وارد کنید
\name{عارف}
% نام خانوادگی دانشجو را وارد کنید
\surname{یزدخواستی}
% شماره دانشجویی دانشجو را وارد کنید
\studentID{4013644018}
% تاریخ پایان‌نامه را وارد کنید
\thesisdate{فروردین 1404}
% به صورت پیش‌فرض برای پایان‌نامه‌های کارشناسی تا دکترا به ترتیب از عبارات «پروژه»، «پایان‌نامه» و «رساله» استفاده می‌شود؛ اگر  نمی‌پسندید هر عنوانی را که مایلید در دستور زیر قرار داده و آنرا از حالت توضیح خارج کنید.
%\projectLabel{پایان‌نامه}

% به صورت پیش‌فرض برای عناوین مقاطع تحصیلی کارشناسی تا دکترا به ترتیب از عبارت «کارشناسی»، «کارشناسی ارشد» و «دکتری» استفاده می‌شود؛ اگر نمی‌پسندید هر عنوانی را که مایلید در دستور زیر قرار داده و آنرا از حالت توضیح خارج کنید.
%\degree{}
%%%%%%%%%%%%%%%%%%%%%%%%%%%%%%%%%%%%%%%%%%%%%%%%%%%%
%% پایان‌نامه خود را تقدیم کنید! %%
\dedication
{
	{\Large تقدیم به:}\\
	\begin{flushleft}{
			\huge
			خانواده که همواره پشتم و همه دردهایم را مرهم هستند\\
		}
	\end{flushleft}
}
%% متن قدردانی %%
%% ترجیحا با توجه به ذوق و سلیقه خود متن قدردانی را تغییر دهید.
\acknowledgement{
	سپاس و آفرین خداوندگار جان آفرین راست ، اوی که آدمی را به گوهر خرد آراست.
	
	در آغاز دستان پدر و مادر نازنینم را به پاس مهر بیکرانشان به گرمی می فشارم، 
	و از استاد راهنما خود سرکار خانم دکتر افسانه فاطمی بابت زمان و انرژی‌ای که گذاشتند سپاس گزاری می‌کنم.
	
	
	و در پایان، سپاس گزاری می کنم از همه اعضای خانواده دانشکده مهندسی کامپیوتر اصفهان به ویژه دوستانم که بهترین روز های زندگی من را رقم زدند.
}
%%%%%%%%%%%%%%%%%%%%%%%%%%%%%%%%%%%%
%چکیده پایان‌نامه را وارد کنید
\fa-abstract{
در این پژوهش، به طراحی و پیاده‌سازی روشی جهت بهبود سیستم های گفتگوی وظیفه‌محور هوشمند تحت عنوان "MindMeld" پرداخته شده است که با ترکیب نوآورانه تکنیک‌های شخصی‌سازی، نمایه‌سازی کاربر و رعایت حریم خصوصی چالش‌های اساسی در حوزه سیستم‌های تعاملی انسان و هوش مصنوعی مورد بررسی و حل قرار گرفته است. این سیستم با بهره‌گیری از تنظیم سریع، فیلتر مشارکتی مبتنی بر آیتم و تحلیل احساسات به طور مؤثر مشکلاتی همچون شروع سرد و محدودیت‌های داده‌های کاربر را مدیریت کرده و تعاملات شخصی‌سازی‌شده‌ای را فراهم می‌کند که منجر به افزایش رضایت کاربر و عملکرد سیستم شده است . 

یکی از ویژگی‌های این پژوهش، پیاده‌سازی حق فراموشی در سیستم گفتگوی وظیفه‌محور است که به کاربران اجازه می‌دهد داده‌های خود را به صورت پویا حذف کنند، بدون اینکه عملکرد کلی سیستم تحت تأثیر قرار گیرد. این رویکرد نه تنها اعتماد کاربر را افزایش می‌دهد، بلکه با قوانین جهانی حفظ حریم خصوصی هماهنگ است. علاوه بر این، ترکیب یادگیری چندشات و تنظیم سریع به سیستم اجازه می‌دهد تا حتی در سناریوهایی که داده‌های محدودی وجود دارد، عملکرد بالایی از خود نشان دهد.

ارزیابی عملکرد سیستم با استفاده از معیارهای جامعی مانند گیجی، تمایز، میزان موفقیت، نرخ تکمیل کار و امتیاز تعامل کاربر انجام شده است. نتایج نشان می‌دهد که این سیستم در مقایسه با رویکردهای سابق، بهبودهایی در معیارهای مختلف داشته است. به عنوان مثال، میزان موفقیت کار در این سیستم نسبت به سیستم‌های مشابه حدود هشت الی هفده درصد افزایش یافته است. 

این پژوهش ضمن معرفی رویکردی مناسب در حوزه سیستم‌های گفتگوی وظیفه‌محور، زمینه‌ساز توسعه سیستم‌هایی است که علاوه بر ارائه خدمات شخصی‌سازی‌شده، به حفظ حریم خصوصی و اخلاق هوش مصنوعی نیز توجه ویژه دارند. در پایان، پیشنهدهایی برای تحقیقات آینده ارائه شده است که شامل بهبود پروفایل‌های کاربری پیشرفته‌تر، گسترش چارچوب‌های ارزیابی و ادغام مدل‌های پیشرفته‌تر زبانی است. 

}
% کلمات کلیدی پایان‌نامه را وارد کنید
\keywords{سیستم گفتگوی وظیفه‌محور، پروفایل کاربری، حق فراموشی، تنظیم سریع، فیلتر مشارکتی}
% انتهای وارد کردن فیلد‌ها
%%%%%%%%%%%%%%%%%%%%%%%%%%%%%%%%%%%%%%%%%%%%%%%%%%%%%%
