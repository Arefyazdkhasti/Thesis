% !TeX root=../main.tex
% در این فایل، عنوان پایان‌نامه، مشخصات خود، متن تقدیمی‌، ستایش، سپاس‌گزاری و چکیده پایان‌نامه را به فارسی، وارد کنید.
% توجه داشته باشید که جدول حاوی مشخصات پروژه/پایان‌نامه/رساله و همچنین، مشخصات داخل آن، به طور خودکار، درج می‌شود.
%%%%%%%%%%%%%%%%%%%%%%%%%%%%%%%%%%%%
% دانشگاه خود را وارد کنید
\university{دانشگاه اصفهان}
% دانشکده، آموزشکده و یا پژوهشکده  خود را وارد کنید
\faculty{دانشکده مهندسی کامپیوتر}
% گروه آموزشی خود را وارد کنید (در صورت نیاز)
\department{گروه مهندسی نرم افزار}
% رشته تحصیلی خود را وارد کنید
\subject{مهندسی نرم افزار}
% گرایش خود را وارد کنید

% عنوان پایان‌نامه را وارد کنید
\title{ارتقای سيستم گفتگوی وظيفه گرا از طریق تنظيم سریع و نماگر سازی کاربر}
% نام استاد راهنما را وارد کنید
\firstsupervisor{دکتر افسانه فاطمی}
\firstsupervisorrank{استاد}

% نام داور خود را وارد نمایید.
\internaljudge{دکتر داور داخلی}
\internaljudgerank{دانشیار}

% نام دانشجو را وارد کنید
\name{عارف}
% نام خانوادگی دانشجو را وارد کنید
\surname{یزدخواستی}
% شماره دانشجویی دانشجو را وارد کنید
\studentID{4013644018}
% تاریخ پایان‌نامه را وارد کنید
\thesisdate{فروردین 1404}
% به صورت پیش‌فرض برای پایان‌نامه‌های کارشناسی تا دکترا به ترتیب از عبارات «پروژه»، «پایان‌نامه» و «رساله» استفاده می‌شود؛ اگر  نمی‌پسندید هر عنوانی را که مایلید در دستور زیر قرار داده و آنرا از حالت توضیح خارج کنید.
%\projectLabel{پایان‌نامه}

% به صورت پیش‌فرض برای عناوین مقاطع تحصیلی کارشناسی تا دکترا به ترتیب از عبارت «کارشناسی»، «کارشناسی ارشد» و «دکتری» استفاده می‌شود؛ اگر نمی‌پسندید هر عنوانی را که مایلید در دستور زیر قرار داده و آنرا از حالت توضیح خارج کنید.
%\degree{}
%%%%%%%%%%%%%%%%%%%%%%%%%%%%%%%%%%%%%%%%%%%%%%%%%%%%
%% پایان‌نامه خود را تقدیم کنید! %%
\dedication
{
	{\Large تقدیم به:}\\
	\begin{flushleft}{
			\huge
			خانواده که همواره پشتم و همه دردهایم را مرهم هستند\\
		}
	\end{flushleft}
}
%% متن قدردانی %%
%% ترجیحا با توجه به ذوق و سلیقه خود متن قدردانی را تغییر دهید.
\acknowledgement{
خدای را سپاس که دانش و آگاهی را به انسان عطا کرده و او را به خرد و نیکی هدایت فرموده است.

در ابتدای این راه، با وجود تمام تلاش‌ها و کوشش‌های شخصی، بدون حمایت بی‌چشمداشت و محبت فراوان والدین گرامی‌ام، پیشرفت و موفقیت برای من میسر نبود.
پدر و مادر عزیزم، که همواره با شکیبایی و دلسوزی در کنار من بودند و با حمایت‌های مادی و معنوی‌شان مسیر سخت زندگی را برایم آسان کردند، به شما افتخار می‌کنم و از صمیم قلب تشکر می‌کنم. این رساله و تمام دستاوردهایم، بیشتر از هر چیز، ثمره زحمات بی‌وقفه و دل‌سوزی‌های بی‌پایان شماست.
	
همچنین، از استاد راهنمای خود، سرکار خانم دکتر افسانه فاطمی، که با حوصله و راهنمایی‌های ارزشمندشان مرا در این مسیر یاری کردند، صمیمانه سپاسگزاری می‌کنم.
	
در پایان، از تمام دوستان عزیز و همراهان همیشگی‌ام که در لحظات سخت و شاد زندگی دانشجویی در کنارم بودند، قدردانی می‌کنم. آن‌ها که با حضور دوستانه‌شان، روزهای سخت را شیرین‌تر و راه را روشن‌تر کردند.
	
این مسیر بدون حضور همه شما میسر نبود. از هر کسی که به نحوی در این مسیر نقشی ایفا کرد، صمیمانه سپاسگزارم.	
}

%%%%%%%%%%%%%%%%%%%%%%%%%%%%%%%%%%%%
%چکیده پایان‌نامه را وارد کنید
\fa-abstract{

سیستم‌های گفتگوی وظیفه‌گرا% 
\LTRfootnote{Task-oriented dialogue system (TODS)}
نقش محوری در ارائه خدمات شخصی‌سازی‌شده از طریق تعاملات انسان و هوش مصنوعی ایفا می‌کنند. با این حال، این سیستم‌ها با چالش‌های مهمی از جمله شخصی‌سازی ناکافی، مشکل شروع سرد%
\LTRfootnote{Cold start}
 و نگرانی‌های مربوط به حریم خصوصی مواجه هستند. این مسائل مانع از توانایی آنها در ارائه تجربیات کاربری یکپارچه و قابل اعتماد می‌شود. علاوه بر این، داده‌های محدود کاربر اغلب دشواری ایجاد تعاملات شخصی‌سازی‌شده مؤثر را تشدید می‌کند، در حالی که عدم رعایت مقررات جهانی حریم خصوصی، استقرار آنها را پیچیده‌تر می‌کند.
برای پرداختن به این چالش‌ها، ما یک رویکرد جدید تحت عنوان "MindMeld" پیشنهاد می‌کنیم که تکنیک‌های پیشرفته‌ای را برای شخصی‌سازی، نماگرکاربر%
\LTRfootnote{User profiling}
 و حفظ حریم خصوصی ادغام می‌کند. این سیستم از تنظیم سریع%
\LTRfootnote{Prompt-tuning}
 مدل‌های زبانی بزرگ برای انطباق سریع با نیازهای خاص کاربر، حتی در سناریوهایی با داده‌های محدود، بهره می‌برد. این سیستم شامل فیلتر مشارکتی مبتنی بر آیتم%
\LTRfootnote{Item-based collaborative filtering}
 است که با تحلیل احساسات%
\LTRfootnote{Semantic analysis}
 بهبود یافته است تا بر مشکل شروع سرد غلبه کند و دقت توصیه را بهبود بخشد. علاوه بر این، MindMeld حق فراموش شدن%
\LTRfootnote{The right to be forgotton}
 را پیاده‌سازی می‌کند و کاربران را قادر می‌سازد تا داده‌های خود را بدون به خطر انداختن عملکرد سیستم، به صورت پویا حذف کنند. این امر انطباق با استانداردهای جهانی حریم خصوصی را تضمین می‌کند و اعتماد کاربر را افزایش می‌دهد. با ترکیب یادگیری چندشات%
\LTRfootnote{Few-shot learning}
 با تنظیم سریع، سیستم حتی در محیط‌های با کمبود داده، به عملکرد قوی دست می‌یابد.
اثربخشی MindMeld با استفاده از معیارهای جامع، از جمله پیچیدگی%
\LTRfootnote{Perplexity}
، تمایز%
\LTRfootnote{Distinct}
، میزان موفقیت%
\LTRfootnote{Success rate}
، نرخ تکمیل کار%
\LTRfootnote{Completion rate}
 و امتیاز تعامل کاربر%
\LTRfootnote{User engagement score}
ارزیابی شد. نتایج، پیشرفت‌هایی را نسبت به رویکردهای موجود نشان می‌دهد. به طور خاص، میزان موفقیت در مقایسه با مطالعات مشابه ۱۷٪ افزایش یافته و تمایز ۱۲٪ بهبود یافته است. این یافته‌ها، توانایی سیستم را در ارائه تعاملات شخصی‌سازی شده و کارآمد، ضمن حفظ حریم خصوصی و اعتماد کاربر، برجسته می‌کند. به طور کلی، MindMeld یک راه حل مقیاس‌پذیر و اخلاقی برای سیستم‌های گفتگوی وظیفه محور ارائه می‌دهد و شکاف‌هایی در شخصی‌سازی، حریم خصوصی و قابلیت استفاده را پر می‌کند.

% کلمات کلیدی پایان‌نامه را وارد کنید
}
\keywords{سیستم گفتگوی وظیفه‌گرا، نماگر کاربری، حق فراموشی، تنظیم سریع، فیلتر مشارکتی}



%%%%%%%%%%%%%%%%%%%%%%%%%%%%%%%%%%%%
% %چکیده پایان‌نامه را وارد کنید
\en-abstract{
\begin{flushleft}
\begin{LTR}

Task-oriented dialogue systems play a central role in delivering personalized services through human-AI interactions. However, these systems face significant challenges, including insufficient personalization, the cold start problem, and privacy concerns. These issues hinder their ability to deliver seamless and reliable user experiences. Furthermore, limited user data often exacerbates the difficulty of creating effective personalized interactions, while lack of compliance with global privacy regulations complicates their deployment. To address these challenges, we propose an approach, named "MindMeld,” that integrates advanced techniques for personalization, user profiling, and privacy. The system leverages prompt-tuning of large language models to quickly adapt to specific user needs, even in data-limited scenarios. The system includes item-based collaborative filtering, enhanced with sentiment analysis to overcome the cold start problem and improve recommendation accuracy. In addition, MindMeld implements the right to be forgotten, enabling users to dynamically delete their data without compromising system performance. This ensures compliance with global privacy standards and increases user trust. By combining few-shot learning with rapid tuning, the system achieves robust performance even in data-poor environments. The effectiveness of MindMeld was evaluated using comprehensive metrics, including Perplexity, Distinct, Success rate, Completion rate, and User engagement score. The results show improvements over existing approaches. Specifically, the success rate increased by seventeen percent and Distinction improved by twelve percent compared to similar studies. These findings highlight the system’s ability to deliver personalized and efficient interactions while maintaining user privacy and trust. Overall, MindMeld provides a scalable and ethical solution for task-based conversational systems, bridging gaps in personalization, privacy, and usability.

\vspace{1cm}
\textbf{Keywords:}  Task-oriented dialogue system (TODS), User Profile, Right To Be Forgotten, Prompt-tuning, Collaborative Filtering.


\end{LTR}
\end{flushleft}
}




% انتهای وارد کردن فیلد‌ها
%%%%%%%%%%%%%%%%%%%%%%%%%%%%%%%%%%%%%%%%%%%%%%%%%%%%%%

