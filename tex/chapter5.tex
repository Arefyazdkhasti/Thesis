% !TeX root=../main.tex
\chapter{[جمع بندی و پیشنهادهایی برای ادامه پژوهش]}
%\thispagestyle{empty} 
\section{مقدمه}

\section{اهداف تحقیق}
هدف این رساله، بهبود سیستم‌های گفتگوی وظیفه‌محور با مقابله با چالش‌ها در شخصی‌سازی، نمایه‌سازی کاربر، مشکل شروع سرد و در عین حال تضمین حریم خصوصی از طریق رعایت حق فراموشی بود. 

با استفاده از تکنیک هایی مانند تنظیم سریع، فیلتر مشارکتی، و پروفایل پویا کاربری، پیشرفت‌های قابل‌توجهی از نظر دقت سیستم، سازگاری و رضایت کاربر به دست آمد.

\subsection{نتایج کلیدی}

\subsubsection{پیشرفت در شخصی سازی}
\begin{itemize}
\item
تولید دیالوگ تطبیقی:
 
سیستم گفتگو با ترکیب ترجیحات کاربر مانند ژانرهای مورد علاقه، تعاملات گذشته و ویژگی‌های خاص زمینه، پاسخ‌های خود را به صورت پویا تنظیم می‌کند. به عنوان مثال کاربرانی که فیلم‌های علمی تخیلی را ترجیح می‌دهند، توصیه‌ها و تعاملات غنی‌شده با زبان مرتبط را دریافت کردند که به طور قابل‌توجهی امتیاز تعامل آنها را بهبود بخشید.

همچنین نمرات گیجی و تمایز بهبود‌یافته تعادل بین ارتباط و تنوع پاسخ را نشان می‌دهد.
\item
 تنظیم سریع برای شخصی‌سازی:
 این رویکرد اجازه کنترل دقیق بر پاسخ‌های سیستم را بدون آموزش مجدد کل مدل می‌دهد. همچنین با استفاده از این روش می‌توان به قابلیت‌های یادگیری چند شات دست یافت و سیستم را قادر ساخت حتی با داده‌های محدود به خوبی تعمیم یابد.
استفاده از این روش، از روش‌های تنظیم دقیق سنتی با کاهش سربار محاسباتی و در عین حال حفظ تعاملات با کیفیت بالا و مرتبط با زمینه، عملکرد بهتری داشت.
\end{itemize}
\subsubsection{حل مشکل شروع سرد}
سیستم با استفاده نوآورانه از فیلتر مشارکتی تعاملات اولیه با کاربران با استفاده از فیلتر اشتراکی مبتنی بر آیتم را افزایش داده‌ها از کاربران مشابه برای شخصی‌سازی به طرز محسوسی افزایش یافت.

در نتیجه سیستم به طور موثر مشکل شروع سرد را برطرف کرد و توصیه‌های دقیق و تعاملات معناداری را برای کاربران جدید ارائه کرد.

\subsubsection{متریک و ارزیابی}
 معیارهای ارزیابی مختلفی مورد بحث قرارگرفت. این سیستم با استفاده از معیارهایی مانند گیجی، تمایز، میزان موفقیت، نرخ تکمیل کار و میزان درگیری کاربر به شدت مورد ارزیابی قرار گرفت.
\begin{itemize}
\item
 گیجی: نمرات پایین‌تر توانایی بهبودیافته سیستم را برای ایجاد پاسخ‌های منسجم و مرتبط با زمینه نشان می‌دهد.
\item
 تمایز: افزایش قابل‌بیانی را نشان داد که نشان‌دهنده تنوع بیشتر در پاسخ‌های ایجاد شده است.
\item
 امتیاز تعامل کاربر : با انطباق پویا با پروفایل‌های کاربر و نرخ تکمیل کار بهبود یافته‌است.
\item
 میزان موفقیت کار: در میان پرس‌وجوهای مختلف، این سیستم در مقایسه با سیستم‌های گفتگوی وظیفه‌محور ذکر شده، هشت الی 17 درصد پیشرفت در موفقیت کار به دست آورد. تجربه کاربری شخصی‌سازی‌شده به طور قابل‌توجهی به این موفقیت کمک کرد.
\end{itemize}
\subsection{ادغام حق فراموشی و حریم خصوصی}
این سیستم نگرانی‌های فزاینده در مورد حفظ حریم خصوصی در حوزه‌ي داده‌ها را از طریق رعایت حق فراموشی برطرف کرد.

 حذف پویا اطلاعات کاربر مکانیسم‌هایی را برای کاربران به منظور درخواست حذف داده‌های خود پیاده‌سازی کرده است، که از عدم وجود اثر باقی‌مانده در پروفایل‌های کاربر یا سیستم‌های فیلتر مشترک اطمینان حاصل می‌کند. این ویژگی اعتماد کاربر را افزایش داد و با قوانین حفظ حریم خصوصی جهانی هماهنگ شد و معیاری برای سیستم‌های هوش مصنوعی اخلاقی تعیین کرد. علیرغم حذف داده‌های کاربر، تنظیم سریع و یادگیری چند شات به سیستم اجازه می‌دهد تا عملکرد بالایی را بدون اتکا به داده‌های تاریخی حفظ کند.

\subsection{سازگاری با دامنه های جدید}

 انعطاف‌پذیری در استفاده از پایگاه دانش برای سیستم فوق نیز قابل بیان است.

 می‌توان از قابلیت انطباق مدل‌های زبانی با هر پایگاه دانشی استفاده کرد و از ارتباط سیستم در حوزه‌های مختلف اطمینان حاصل کرد. در صورت وجود دیتاست‌ مرتبط با هرحوزه، سیستم را می‌توان به راحتی به برنامه‌های کاربردی خدمات مشتری، مراقبت‌های بهداشتی و تجارت الکترونیک گسترش داد.

همچین تکنیک‌های شروع سرد و آموزش چندشات، سیستم را قادر می‌سازد حتی در سناریوهایی که قبلاً دیده نشده بود یا با حداقل داده‌های مربوط به کار، به خوبی عمل کند و نیاز به بازآموزی گسترده را به حداقل برساند.

\subsection{زمینه‌های فراگیر}

\begin{itemize}
\item
رضایت و تعامل کاربر

 شخصی‌سازی و رعایت حریم خصوصی به طور جمعی رضایت کاربر را افزایش می‌دهد، همانطور که با نمرات تعامل بالاتر مشهود است.
 تعادل بین حریم خصوصی و عملکرد استاندارد جدیدی را برای سیستم‌های هوش مصنوعی در حوزه‌های تنظیم‌شده ایجاد می‌کند.
\item
مقیاس پذیری

ماهیت سبک تنظیم سریع مقیاس‌پذیری را در دستگاه‌های با قدرت محاسباتی محدود تضمین می‌کند و این رویکرد را برای مخاطبان گسترده‌تری قابل دسترسی می‌سازد.
\item
حفظ حریم خصوصی

گنجاندن انطباق با حق فراموشی، همسویی سیستم را با شیوه‌های هوش‌مصنوعی اخلاقی برجسته می‌کند و به مسائل مربوط به اعتماد کاربر که اغلب در سیستم‌های گفتگوی وظیفه‌محور سنتی نادیده گرفته می‌شوند، می‌پردازد.
\end{itemize}

\section{مشارکت و نوآوری}
این رساله از سیستم‌های گفتگوی وظیفه محور، با تمرکز بر شخصی سازی،شروع سرد و حفظ حریم خصوصی، در حالی که از تکنیک های نوآورانه مانند تنظیم سریع و ساخت پروفایل کاربری با استفاده از تکنیک‌هایی مانند فیلتر مشارکتی مبتنی بر آیتم و تحلیل احساسات استفاده می کند.

در زیر یک طرح کلی ساختار یافته از مشارکت ها و جنبه های بدیع این تحقیق آورده شده است.

\subsection{مشارکت های اصلی}
\subsubsection{شخصی سازی در سیستم های گفتگو}


\begin{itemize}
\item
 پروفایل کاربری پویا:

 یک چارچوب جدید برای ایجاد و حفظ نمایه‌های کاربر پویا با استخراج اولویت‌ها (مانند ژانرهای برتر، فیلم‌های پسندیده) از تعاملات ایجاد شد.

در نتیجه سیستم پاسخ‌ها را در زمان واقعی تطبیق داده و تعامل و رضایت کاربر را بهبود می‌بخشد.

برخلاف سیستم‌های شخصی‌سازی ایستا یا مبتنی‌بر قانون، این رویکرد یادگیری و سازگاری مداوم را امکان‌پذیر می‌کند که پروفایل هرکاربر به مرور و با تعامل با برنامه به‌روز شود.
\item
 تعاملات آگاه از زمینه:

استفاده از تنظیم سریع امکان ادغام بلادرنگ زمینه کاربر (به عنوان مثال، ترجیحات، تاریخچه هر گفتگو) را در پاسخ‌های سیستم فراهم می‌کند. با اتکا به این روش افزایش ارتباط و انسجام در گفتگو ایجاد شده‌است که با نرخ موفقیت بالاتر و معیارهای تعامل این پیشرفت نشان داده شده‌است.
\item
پرداختن به مشکل شروع سرد:

این سیستم علاوه بر ژانرهای دریافتی مورد علاقه کاربر نوپا، فیلتر مشارکتی مبتنی بر آیتم را برای مدیریت مشکل شروع سرد با استفاده از داده‌های کاربران مشابه پیاده‌سازی کرده است.

این رویکرد با موفقیت شکاف را برای کاربران جدید پر می‌کند و از تعاملات معنی‌دار حتی با حداقل داده‌های اولیه اطمینان حاصل می‌ند. این عملکرد جهت حل مشکل شروع سرد در سیستم نشان می‌دهد که فیلتر کردن مشارکتی می‌تواند فراتر از سیستم‌های توصیه گسترش یابد تا شخصی‌سازی گفتگو را بهبود بخشد.
\item
 آموزش چند شات:

 این تکنیک‌ها را برای مدیریت مؤثر سناریوهای نادیده، به حداقل رساندن نیاز به داده‌های گسترده ویژه کار، به کار گرفت.
به همین علت امکان استقرار سریع سیستم های گفتگو در حوزه‌های جدید بدون آموزش مجدد فراهم آمده است.
\end{itemize}


\subsubsection{حریم خصوصی و هوش مصنوعی اخلاقی}

\begin{itemize}
\item
 حق فراموش شدن:

در این رساله مکانیسم‌هایی برای انطباق با حق فراموشی استفاده شد که به کاربران اجازه می‌دهد بدون تأثیر بر عملکرد سیستم، درخواست حذف داده‌های خود را داشته باشند.

این یکی از پیاده‌سازی‌های سیستم‌های گفتگوی وظیفه محور است که انطباق حق فراموشی را با حفظ عملکرد عملیاتی می‌کند. با ایجاد این مهم، اعتماد کاربر جهت استفاده از سیتسم افزایش یافته و با قوانین جهانی حفظ حریم خصوصی داده‌ها هماهنگ می‌شود.
\item
 متعادل کردن حریم خصوصی و شخصی سازی:

روش‌های توسعه‌یافته برای ایجاد تعادل بین حریم خصوصی و شخصی‌سازی با اطمینان از اینکه داده‌های کاربر حذف شده عملکرد کلی سیستم را به خطر نمی‌اندازد. این تعادل امکان شخصی‌سازی حفظ حریم خصوصی در سیستم‌های هوش‌مصنوعی را در کنار قابلیت شخصی‌سازی‌کردن نتایج نشان داد.
\end{itemize}

\subsubsection{معیارها و چارچوب ارزیابی}
معیارهای ارزیابی پیشرفته:

یک چارچوب ارزیابی قوی با استفاده از گیجی، تمایز، میزان موفقیت، میزان تکمیل کار، امتیاز تعامل کاربر و همچنین دقت توصیه شخصی و تطابق تنوع نمایه معرفی شدند.
معیارهای فوق مجموعه ای جامع از معیارها را معرفی کرد که به طور خاص برای ارزیابی سیستم‌های گفتگوی وظیفه محور شخصی سازی شده و سازگار طراحی شده است.این معیارها روشی استاندارد برای ارزیابی عملکرد سیستم‌های گفتگوی وظیفه محور فراتر از معیارهای سنتی ارائه می‌کند.


\subsubsection{تکنیک ها و روش های بدیع}


\begin{itemize}
\item
 تنظیم سریع به عنوان یک روش اصلی:

 یک جایگزین سبک وزن و در عین حال قدرتمند برای تنظیم دقیق، تنظیم سریع برای دستیابی به شخصی‌سازی و سازگاری محوری بود.
 
استفاده از این روش، کارآمدی تنظیم سریع در تنظیم پویا دیالوگ‌ها بدون متحمل‌شدن هزینه‌های محاسباتی بالا را به نمایش گذاشت. همچنین تنظیم سریع به عنوان یک رویکرد قابل دوام برای سیستم‌های گفتگوی وظیفه محور ایجاد شد.

\item
 رویکرد ترکیبی برای شخصی سازی:

 ترکیبی از پروفایل کاربری، فیلترکردن مشارکتی و مدل‌های زبانی برای ایجاد یک سیستم ترکیبی که از نقاط قوت هر جزء استفاده می‌کند. این ترکیب روش‌ها و نتایج ارزیابی به خوبی نشان داد که چگونه مدل‌های ترکیبی می‌توانند از رویکردهای مستقل در شخصی سازی و رضایت کاربر بهتر عمل کنند.

\item
 ادغام هوش مصنوعی اخلاقی:

 به نگرانی‌های مربوط به حریم خصوصی و اخلاقی از طریق راه‌حل‌های نوآورانه پرداخته و استاندارد جدیدی برای شیوه‌های هوش مصنوعی اخلاقی در سیستم‌های گفتگو ایجاد می‌کند.
\end{itemize}


این تحقیق با پرداختن به چالش‌های قدیمی در شخصی‌سازی، سازگاری و رعایت حریم خصوصی، شکاف‌های حیاتی در توسعه سیستم‌های گفتگوی وظیفه‌محور را پر می‌کند. همچنین یک نقشه راه برای ادغام شیوه های هوش‌مصنوعی اخلاقی در سیستم‌های گفتگو ارائه کرد و تنظیم سریع به عنوان یک تکنیک اصلی برای سیستم‌های تطبیقی ​​و مقیاس‌پذیر ایجاد شد.


\section{پیشنهادهایی برای تحقیقات آینده}
این بخش مسیرهای بالقوه ای را برای کاوش بیشتر، بر اساس مبانی و یافته های این تحقیق بیان می‌کند. پیشنهادات زیر زمینه‌هایی را برای بهبود، گسترش و نوآوری در سیستم‌های گفتگوی وظیفه‌محور، شخصی‌سازی و هوش‌مصنوعی اخلاقی برجسته می‌کنند.

\subsection{پیشبرد تکنیک های شخصی سازی}

\subsubsection{پروفایل کاربری پیشرفته}

\begin{itemize}
\item
 پروفایل کاربری عمیق

 تحقیقات آینده می‌تواند تکنیک‌های پیچیده‌تر پروفایل‌سازی کاربر را بررسی کند، و همچین با بهره‌بردن از داده‌های چندوجهی مانند صدا، ویدیو یا فعالیت رسانه‌های اجتماعی کاربر، درک بهتری از ترجیحات کاربر ایجاد کند.
با استفاده از این روش‌ها شخصی‌سازی برای هر کاربر را فراتر از داده‌های صرفا متنی خواهد برد و نتیجه را بهبود می‌بخشد.
\item
 ترجیحات زمانی و متنی

 علاوه بر جنبه‌های متنوع پروفایل ‌کاربری می‌توان از اینکه چگونه ترجیحات کاربر در طول زمان تغییر می کند یا بر اساس زمینه های موقعیتی متفاوت است نیز جهت بهبود و ارتقای پروفایل کاربری استفاده کرد.
به عنوان مثال باید مدل‌های پویا را پیاده‌سازی کرد که از رفتار کاربر در بازه های زمانی مختلف یا در طول سناریوهای خاص یاد می گیرند و ترجیحات وی را به‌روز می‌کنند.
\end{itemize}

\subsubsection{فیلتر مشارکتی و ترکیبی}
می‌توان با کاوش روش‌های ترکیبی که فیلتر مشارکتی را با مدل‌های مبتنی بر یادگیری عمیق ترکیب می‌کند،مشکلات شروع سرد را به طور مؤثرتری مدیریت کرد.به عنوان مثال می‌توان با استفاده از تکنیک های مبتنی بر نمودار ارتباطات پنهان بین کاربران و آیتم ها را به طور موثرتر و بهتر شناسایی کرد.


\subsection{گسترش چارچوب های ارزیابی}

\begin{itemize}
\item
معیارهای جدید

همچنین می‌توان با تعریف معیارهای جدید، اندازه‌گیری میزان درک و پاسخ سیستم به احساسات کاربر را بیشتر و بهتر بررسی کرد. به عنوان مثال تعریف امتیاز درگیری عاطفی بر اساس احساسات و حالات عاطفی در پاسخ‌ها که سیستم با توجه به حال عاطفی کاربر پاسخ‌های متناسب به وی بدهد.
\item
 معیارهای خاص دامنه

 معیارهای ارزیابی برای حوزه‌های خاص (به عنوان مثال، دقت برای توصیه‌های پزشکی، ارتباط با توصیه‌های فیلم). در کارهای آتی می‌توان به تاثیر عمیق معیارهای دامنه خاص بر رضایت کاربر تمرکز کرد و آنرا بهبود داد.
\end{itemize}

\subsection{یکپارچه سازی مدل های پیشرفته}
\begin{itemize}
\item
ارتقاء مدل زبانی

با گنجاندن مدل‌های زبانی پیشرفته‌تر که پتانسیل آن‌ها را برای درک بهتر، تولید و شخصی‌سازی بیشتر است به طور حتمی نتایج بهتری در پی خواهد بود. ولی چالش استفاده از این مدل های زبانی حفظ تعادل هزینه محاسباتی با بهبود عملکرد سیستم خواهد بود.
\item
ادغام چندوجهی

ترکیب روش واکاوی متن با سایر روش‌های سیستم‌های توصیه مانند دیالوگ‌های مبتنی بر تصاویر، صدا و ویدئو برای تعاملات بیشتر با کاربر نیز می‌تواند جزو کارهای آتی قرار بگیرد. به عنوان مثال توصیه فیلم‌های دارای تریلر یا صحنه بر اساس ترجیحات کاربر می تواند در ارایه پیشنهادهای بهتر به کاربر موثر باشد.
\end{itemize}

پیشنهادات ذکر شده در بالا، پتانسیل قابل توجهی را برای توسعه بیشتر در سیستم‌های گفتگوی وظیفه‌محور، شخصی‌سازی و هوش‌مصنوعی حفظ حریم خصوصی نشان می‌دهد. با کاوش در این جهت‌ها، محققان آینده می‌توانند بر روی این کار برای افزایش رضایت کاربر، سازگاری سیستم و استقرار هوش مصنوعی اخلاقی کار کنند.


\section{محدودیت ها و چالش ها}

این بخش به طور انتقادی محدودیت‌ها و مشکلاتی را که در طول توسعه و ارزیابی سیستم گفتگوی وظیفه‌محور با نمایه‌های شخصی‌شده و تنظیم سریع مواجه می‌شود، بررسی می‌کند. محدودیت‌ها حوزه‌هایی را برجسته می‌کنند که بر نتایج تأثیر گذاشتند و بینش‌هایی را در مورد چالش‌هایی که باید برای بهبود بیشتر مورد توجه قرار گیرند، ارائه می‌دهند.

\subsection{محدودیت داده ها}
محدودیت های مجموعه داده به شرح زیر است.
\begin{itemize}
\item
 تولید داده های سفارشی

به دلیل عدم وجود مجموعه داده‌های مکالمه‌ای که از قبل برای توصیه‌های فیلم طراحی شده‌اند، زمان و تلاش قابل توجهی برای ایجاد مجموعه‌ای که داده‌های مکالمه را تقلید می‌کند، صرف شد. داده‌های تولید شده، اگرچه مؤثر هستند، ممکن است فاقد تنوع و عمق مکالمات در دنیای واقعی باشند به دلیل آنکه در نهایت این مکالمات ساختی و با استفاده از قالب و الگوهای از پیش تعیین‌شده هستند که نمی‌توانند با تنوع مکالمات واقعی مقایسه گردد. با بهبود و ارتقای این مجموعه داده، پاسخ‌های متنوع‌تر و با ترکیب کلمات بهتر ایجاد خواهد شد.

\item
 حجم داده محدود برای تنظیم دقیق

حجم داده‌های تولیدشده برای تنظیم دقیق مدل زبانی در مقایسه با مقیاس داده‌های مورد استفاده برای آموزش مدل های زبانی بزرگ محدود بود و همین باعث کاهش تعمیم‌پذیری و سازگاری مدل و عدم اجرای عملیات تنظیم دقیق بر روی مدل شده است.
\end{itemize}

\subsection{محدودیت های محاسباتی}
برای استفاده از مدل‌های زبانی آموزش آنها و اجرای تکنیک‌های مختلف، سخت‌افزار عضو جدایی ناپذیر از این عملیات است. 

\begin{itemize}
\item
محدودیت های منابع سخت افزاری

 اتکا به پردازنده‌های گرافیکی، اندازه مدل، اندازه دسته‌ای و تکرارهای آموزشی را محدود می‌کرد. به عنوان مثال به دلیل داشتن منابع سخت‌افزاری محدودتر ، اجبار به استفاده از مدل‌های زبانی با پارامترهای کمتر به جای مدل های پیشرفته تر مانند جی‌پی‌تی۴ باعث عدم دسترسی به نتایج بهتر شد.

\item
 محدودیت های زمانی

 چرخه‌های تمرین و تنظیم دقیق به دلیل محدودیت‌های زمانی غیرعملی بود که بر بهینه‌سازی و آزمایش هایپرپارامتر تأثیر گذاشت.
\end{itemize}


\subsection{چالش های پروفایل کاربری}
\begin{itemize}
\item
مشکل شروع سرد

برای کاربرانی که تازه با سیتسم شروع با کار می‌کنند، داده‌های تاریخی محدود است و می‌بایست توصیه‌های شخصی‌سازی‌شده را در ابتدا دقیق‌تر کرد تا برای این دسته کاربران نیز نتایج قابل قبولی ارایه شود. با استفاده از فیلتر مشارکتی و استفاده از اولویت‌های پیش‌فرض ژانر تا حدی این مشکل را برطرف شد.

\item
تحول تنظیمات کاربر در مرور زمان اگر در سیستم اعمال نشود باعث کاهش رضایت کاربر خواهد شد. تکنیک‌های پروفایل پویا برای انطباق با تغییر سلیقه کاربر در طول زمان تلاش کردند تا به حل این مشکل بیایند. لذا با ترکیب مدل‌های پروفایل کاربر پویا با قابلیت‌های یادگیری افزایشی و دریافت بازخورد مناسب از کاربر و به‌روزرسانی ترجیحات وی می‌توان از این مشکل جلوگیری کرد.
\end{itemize}

\subsection{نگرانی های حفظ حریم خصوصی}
پیاده سازی حق فراموشی در سیستم‌‌های گفتگو دارای پیچیدگی‌هایی است. اگرچه حق فراموشی اجرا شد، فرآیند حذف ایمن داده‌های کاربر بدون تأثیر بر مدل جهانی نیازمند تلاش قابل توجهی بود. همچنین می‌توان با کاوش در رویکردهای غیرمتمرکز برای افزایش حریم خصوصی داده ها این پیاده‌سازی را به صورت موثرتر پیاده‌سازی کرد و آن‌را بهبود داد.


\subsection{چالش های ارزیابی}
معیارهای ارزیابی سیستم‌های گفتگو به طور کلی ماهیت ذهنی دارند. معیارهایی مانند رضایت کاربر و امتیازات تعامل به بازخورد ذهنی بستگی دارد که ممکن است بین کاربران متفاوت باشد. که این ماهیت ذهنی مشکل در نتیجه گیری قابل اجرا جهانی را در پی خواهد داشت.

همچنین معیارهای مرسوم دیگر نیازمند یک جواب خاص یا جواب طلایی جهت مقایسه جواب نهایی مدل با آن جواب طلایی بودند ولی در سیستم‌های گفتگو که با مدل‌های زبانی کار می‌کنند ممکن است جواب مدل درست باشد ولی از جواب طلایی که مبنای امتیازدهی است متفاوت باشد.

به همین علت باید معیارهای جامع‌تر که هم مجزا از پاسخ طلایی و هم مقداری خارج از ماهیت ذهنی کاربران باشد مانند نرخ تکمیل کار یا نرخ کلیک یا میزان درگیربودن کاربر در مکالمه ایجاد کرد.

\subsection{محدودیت های مدل}
اندازه و معماری مدل با توجه به استفاده از مدل زبانی با پارامترهای کمتر انتخاب شد. در حالی که مدل زبانی مورد استفاده برای محدوده این پروژه کافی بود، عملکرد و درک متنی آن نسبت به مدل های بزرگتر مانند جی‌پی‌تی۴ یا لاما۳ پایین‌تر بود. استفاده از مدل با پارامترهای کمتر منجر به پاسخ‌های کم‌تر و کوتاه‌تر انسجام در نتیجه نتایج ناکارآمدتر خواهدشد.


\subsection{چالش های تنظیم سریع}
پاسخ‌های نهایی مدل و نتایج ارزیابی آن به کیفیت تنظیم سریع بسیار وابسته است. همچنین عملکرد مدل‌های تنظیم‌شده سریع به کیفیت درخواست‌های ورودی بسیار حساس بود. به طور مثال دستورات بد ساخته‌شده منجر به خروجی‌های مدل غیربهینه شد.

علاوه بر آن علیرغم بهبودهایی که از طریق تنظیم سریع انجام شد، این مدل قابلیت‌های محدودی را در سناریوهای شات صفر نشان داد. با استفاده از رویکردهای فرایادگیری برای بهبود سازگاری می‌توان این مشکل را به حداقل رساند.

\subsection{شخصی سازی پاسخ پویا}

تنظیم پویا پاسخ‌ها بر اساس بازخورد کاربر در حال تکامل، چالش‌هایی را در حفظ انسجام و ارتباط پاسخ ایجاد می‌کند.
برای بهبود عملکرد این بخش باید طراحی حلقه های بازخورد بلادرنگ برای پاسخ های تطبیقی بهبود یافته و به طور موثرتر در سیستم پیاده سازی شوند.

محدودیت‌ها و چالش‌هایی که بیان شد، بر پیچیدگی‌های ساخت سیستم‌های گفتگوی وظیفه‌محور با ویژگی‌های شخصی‌سازی و حفظ حریم خصوصی بالا تأکید می‌کند. پرداختن به این محدودیت‌ها در کارهای آینده، توسعه سیستم‌های گفتگوی قوی‌تر، سازگارتر و کاربر محور را ممکن می‌سازد.